\documentclass[12pt, letterpaper, spanish]{article}
\usepackage{babel}
\usepackage[T1]{fontenc}
\usepackage{textcomp}
\usepackage[utf8]{inputenc} % Puede depender del sistema o editor
\usepackage{amsmath}
\usepackage{amsfonts}
\usepackage{amssymb}
\usepackage[left=2cm,right=2cm,top=2cm,bottom=2cm]{geometry}
\usepackage{mathrsfs}
\usepackage{pstricks-add}
\usepackage{pgfkeys}
\usepackage{setspace}
\usepackage{fancyhdr}
\usepackage{graphicx}
\usepackage{enumerate}

\usepackage{tikz}
\usetikzlibrary{fit,positioning}

\begin{document}
\begin{titlepage}
	\centering
	{\scshape\LARGE Instituto Politécnico Nacional\\ Unidad Profesional Interdisiplinaria de Ingenierias campus Zacatecas\par}
	\vspace{1cm}
	{\scshape\Large Probabilidad Y Estadistica\par}
	\vspace{1.5cm}
	{\huge\bfseries Tarea 4\par}
	\vspace{2cm}
	{\Large\itshape Gerardo Ayala Juárez\par}
	{\Large\itshape Olando Odiseo Belmonte Flores\par}
	{\Large\itshape Lucía Monserrat López Méndez\par}
	{\Large\itshape Oscar Iván Palacios Ulloa\par}
	\vfill
	Maestro:\par
	\textsc{
	Rosendo Vasquez Bañuelos}
	\vfill
% Bottom of the page
	{\large \today \par}
\end{titlepage}

\textbf{46.} Suponga que un individuo es seleccionado al azar de la población de todos los adultos varones que viven en Estados Unidos. Sea $A$ el evento en que el individuo seleccionado tiene una estatura de más de 6 pies y sea $B$ el evento en que el individuo seleccionado es un jugador profesional de básquetbol. ¿Cuál piensa que es más grande, $P(A | B)$ o $P(B | A)$? ¿Por qué? \\
R= $P(B|A)$, por que el básquetbol es un deporte cullos ejercicios de entrenamiento tienden a hacer que los practicantes se estiren\\\\

\textbf{48.} Re considere la situación del sistema defectuoso descrito en el ejercicio 26 (Tarea 3).\\

\begin{enumerate}[a)]
    \item Dado que el sistea tiene un defecto de tipo 1, ¿cuál es la probabilidad de que tenga un defecto de tipo 2?\\
    $P(A_2|A_1)=\displaystyle\frac{P(A_1\cap A_2)}{P(A_1)}=\displaystyle\frac{P(A_1)+P(A_2)-P(A_1\cup A_2)}{P(A_1)}=\displaystyle\frac{0.12+0.07-0.13}{0.12}=\displaystyle\frac{0.06}{0.12}=0.5$
    \item Dado que el sistema tiene un defecto de tipo 1, ¿cuál es la probabilidad de que tenga los tres tipos de defectos?\\
    $P(A_2\cap A_2|A_1)=\displaystyle\frac{P(A_1\cap A_2\cap A_3)}{P(A_1)}=\displaystyle\frac{0.01}{0.12}=0.083$
    \item Dado que el sistema tiene por lo menos un tipo de defecto, ¿Cual es la probabilidad de que tenga exactamente un tipo de defecto?\\
    $P(A_1\cap A_2)=0.12+0.07-0.13=0.06$\\
    $P(A_1\cap A_3)=0.12+0.05-0.14=0.03$\\
    $P(A_2\cap A_3)=0.05+0.07-0.10=0.02$\\
    $P(A_1\cup A_2\cup A_3)=0.12+0.07+0.05-0.06 -0.03-0.02+0.01=0.14$\\\\
    $P(A_1^c)=1-0.12=0.88$\\
    $P(A_2^c)=1-.07=0.93$\\
    $P(A_3^c)=1-0.05=0.95$\\

    $P(A_1^c\cap A_2^c)=P((A_1\cup A_2)^c)=1-0.13=0.87$\\
    $P(A_1^c\cap A_3^c)=P((A_1\cup A_3)^c)=1-0.14=0.86$\\
    $P(A_2^c\cap A_3^c)=P((A_2\cup A_3)^c)=1-0.1=0.9$\\
    $P((A_1^c\cap A_2^c)\cup (A_1^c\cap A_3^c)\cup (A_2^c\cap A_3^c))=P(A_1^c\cap A_2^c)+P(A_1^c\cap A_3^c)+P(A_2^c\cap A_3^c)+2P((A_1\cup A_2\cup A_3)^c)=0.87+0.86+0.9+2(0.86)=0.91$\\
    $P((A_1\cup A_2\cup A_3)\cap ((A_1^c\cap A_2^c)\cup (A_1^c\cap A_3^c)\cup (A_2^c\cap A_3^c)))=P((A_1^c\cap A_2^c)\cup (A_1^c\cap A_3^c)\cup (A_2^c\cap A_3^c))-P((A_1\cup A_2\cup A_3)^c)=0.91-0.86=0.05$\\

    $P((A_1\cup A_2\cup A_3)\cap ((A_1^c\cap A_2^c)\cup (A_1^c\cap A_3^c)\cup (A_2^c\cap A_3^c))|A_1\cup A_2\cup A_3)=$\vskip.05cm $\displaystyle\frac{(A_1\cup A_2\cup A_3)\cap ((A_1^c\cap A_2^c)\cup (A_1^c\cap A_3^c)\cup (A_2^c\cap A_3^c))}{A_1\cup A_2\cup A_3}=\displaystyle\frac{0.05}{0.14}=0.35$
    \item Dado que el sistema tiene los primeros dos tipos de defectos, ¿cuál es la probabilidad de que no tenga el tercer tipo de defecto?\\
    $P(A_3^c|A_1\cap A_2)=\displaystyle\frac{P(A_1\cap A_2\cap A_3^c)}{P(A_1\cap A_2)}=\displaystyle\frac{P(A_1\cap A_2)-P(A_1\cap A_2\cap A_3)}{P(A_1\cap A_2)}=\\ \displaystyle\frac{0.06-0.01}{0.06}=\displaystyle\frac{0.05}{0.06}=0.83$
\end{enumerate}
\vskip 0.1 cm
\textbf{50.} Una tienda de departamentos vende camisa sport en tres tallas (chica, mediana y grande), tres diseños (a cuadros, estampadas y a rayas) y dos largos de manga (larga y corta). Las tablas adjuntas dan las proporciones de camisas vendidas en las combinaciones de categoría.\\

\textbf{Manga corta}\\
\begin{tabular}{c}
    \hline
    Diseño \\
    \begin{tabular}{cccc}
        \hline
        Talla & Cuadros & Estampada & Rayas\\ \hline
        CH & 0.04 & 0.02 & 0.05\\
        M & 0.08 & 0.07 & 0.12\\
        G & 0.03 & 0.07 & 0.08
    \end{tabular}
\end{tabular}\\

\textbf{Manga Larga}\\
\begin{tabular}{c}
    \hline
    Diseño \\
    \begin{tabular}{cccc}
        \hline
        Talla & Cuadros & Estampada & Rayas\\ \hline
        CH & 0.03 & 0.02 & 0.03\\
        M & 0.10 & 0.05 & 0.07\\
        G & 0.04 & 0.02 & 0.08
    \end{tabular}
\end{tabular}

\begin{enumerate}[a)]
    \item ¿Cuál es la probabilidad de que la siguiente camisa vendida sea una camisa mediana estampada de manga larga?\\
    R=0.05
    \item ¿Cuál es la probabilidad de que la siguiente camisa vendida sea una camisa estampada mediana?\\
    R=0.07+0.05=0.12
    \item ¿Cuál es la probabiliad de que la siguiente camisa vendida sea de manga corta?¿De ,anga larga?\\
    $A:\ se\ vende\ una\ camisa\ de\ manga\ corta$\\
    $A^c:\ se\ vende\ una\ camisa\ de\ manga\ larga$\\
    $P(A)=0.04+0.08+0.03+0.02+0.07+0.07+0.05+0.12+0.08=0.56$\\
    $P(A^c)=1-P(A)=1-.56=0.44$
    \item ¿Cuál es la probabilidad de que la talla de la siguiente camisa vendida sea mediana?¿Que la siguiente camisa vendida sea estampada?\\
    $B:\ Se\ vebde\ una\ camisa\ de\ talla\ meiana$\\
    $C:\ Se\ vende\ una\ camisa\ estampada$\\
    $P(B)=0.08+0.07+0.12+0.1+0.05+0.07=0.49$\\
    $P(C)=0.02+0.07+0.07+0.02+0.05+0.02=0.25$
    \item Dado que la camisa que se acaba de vender era de manga corta a cuadros, ¿cuál es la probabilidad de que fuera mediana\\
    $\displaystyle\frac{0.08}{0.04+0.08+0.03}=\displaystyle\frac{0.08}{0.15}=0.53$
    \item Dado que la camisa que se acaba de vender era mediana a cuadros, ¿cuál es la probabilidad de que fuera manga corta?¿De manga larga?\\
    $D:\ Se\ vende\ una\ camisa\ mediana\ a\ cuadros$\\
    $P(A|D)=\displaystyle\frac{P(A\cap D)}{P(D)}=\displaystyle\frac{0.08}{0.08+0.1}=\displaystyle\frac{0.08}{0.18}=0.44$\\
    $P(A^c|D)=\displaystyle\frac{P(A^c\cap D)}{P(D)}=\displaystyle\frac{0.1}{0.08+0.1}=\displaystyle\frac{0.1}{0.18}=0.55$
\end{enumerate}

\textbf{52.} Un sistema se compone de bombas idénticas, \#1 y \#2. Si una falla, el sistema seguira operando. Sin embargo, debido al esfuerzo adicional, ahora es más probable que la bomba restante falle de lo que era originalmente. Es decir $r=P(\#2\ falla|\#1\ falla)>P(\#2\ falla)=q$. Si por lo menos una bomba falla al rededor del final de su vida útil en $7\%$ de todos los sistemas	y ambas bombas fallan sólo en $1\%$, ¿cuál es la probabilidad de que la bomba \#1 falle durante su vida útil de diseño?\\
$A:\ Falla\ por\ lo\ menos\ una\ bomba$
$B:\ Fallan\ ambas\ bombas$\\
$P(#1\ falle)= A\cup B$\\
$P(#1\ falle)=0.07+0.01=0.08$\vskip0.5cm

\textbf{54.} En el ejercicio 13, $A_i={proyecto\ otrogado\ \ i}$, con $i=1,2,3$. Use las probabilidades dadas allí para calcular las siguientes probabilidades y explique en palabras el significado de cada una.

\begin{enumerate}[a)]
	\item $P(A_2|A_1)$\\
	La probabilidad de que le otorgen el proyecto a $A_2$ dado que se lo otorgaron a $A_1$\\
	$P(A_2|A_1)=\displaystyle\frac{P(A_1\cap A_2)}{P(A_1)}=\displaystyle\frac{0.11}{0.22}=0.5$
	\item $P(A_2\cap A_3|A_1)$\\
	Probabilidad de que le entreguen e proyecto a $A_2$ y $A_3$ dado que ya se lo entregaron a $A_1$\\
	$P(A_2\cap A_3|A_1)=\displaystyle\frac{P(A_1\cap A_2\cap A_3)}{P(A_1)}=\displaystyle\frac{0.01}{0.11}=0.09$
	\item $P(A_2\cup A_3|A_1)$\\
	Probabilidad de que le otorgen el proyecto a $A_2$ o $A_3$ dado que ya se lo otorgaron a $A_1$\\
	$P(A_2\cup A_3|A_1)=\displaystyle\frac{P((A_2\cup A_3)\cap A_1)}{P(A_1)}$
	\item $P(A_1\cap A_2\cap A_3|A_1\cup A_2\cup A_3)\displaystyle\frac{0.15}{0.22}=0.68$\\
	Probabilidad de que le den el proyecto a las dado que se lo dan por lo menos a uno\\
	$P(A_1\cap A_2\cap A_3|A_1\cup A_2\cup A_3)=\displaystyle\frac{P(A_1\cap A_2\cap A_3)}{P(A_1\cup A_2\cup A_3)}=\displaystyle\frac{0.01}{0.53}=0.018$
\end{enumerate}\vskip0.5cm

\textbf{56.} Para los eventos $A$ y $B$ con $P(B)>0$, demuestre que $P(A|B)+P(A^c|B)=1$\\\\
$P(A|B)+P(A^c|B)=\displaystyle\frac{P(A\cap B)}{P(B)}+\displaystyle\frac{P(A^c\cap B)}{P(B)}=\displaystyle\frac{P(A\cap B)+P(A^c\cap B)}{P(B)}=\displaystyle\frac{P(B)}{P(B)}=1$\vskip0.5cm

\textbf{58.} Demuestre que para tres eventos cualesquiera $A$, $B$ y $C$ con $P(C)>0$, $P(A\cup B|C)=P(A|C)+P(B|C)-P(A\cap B|C)$\\\\
$P(A|C)+P(B|C)-P(A\cap B|C)=\displaystyle\frac{P(A\cap C)}{P(C)}+\displaystyle\frac{P(B\cap C)}{P(C)}-\displaystyle\frac{P(A\cap B\cap C)}{P(C)}=\vskip0.20cm \displaystyle\frac{P(A\cap C)+P(B\cap C)-P(A\cap B\cap C)}{P(C)}=\displaystyle\frac{(P(A)+P(B)-P(A\cap B))\cap P(C)}{P(C)}\vskip0.5cm =\displaystyle\frac{P((A\cup B)\cap C)}{P(C)}=P(A\cup B|P(C))$
\vskip0.5cm

\textbf{60.} El $70\%$ de las aeronaves ligeras que desaparecen en vuelo en cierto país son posterior mente localizadas. De las aeronaves localizadas, $60\%$ cuentan con un localizador de emergencia, mientras que $90\%$ de las aeronaves no localizadas no cuentan con dicho localizador. Suponga que una aeronave ligera ha desaparecido.

\begin{enumerate}[a)]
	\item Si tiene un localizador de emergencia, ¿cuál es la probabilidad de que no sea localizada?\\
	$A:\ Aero\ nave\ perdida\ localizada$\\
	$A^c:\ Aero\ nave\ perdida\ no\ localizada$\\
	$B:\ Localizada\ con\ localizador$\\
	$B^c:\ Localizada\ sin\ localizador$\\\\
	\item Si no tiene un localizador de emergencia, ¿cuál es la probabilidad de que no sea localizada?
\end{enumerate}\vskip0.5cm

\textbf{62.} Una compañía que fabrica cámaras de video produce un modelo básico y un modelo de lujo. Durante el año pasado, $40\%$ de las cámaras vendidas fueron del modelo básico. De aquellos que compraron el modelo básico, $30\%$ adquirieron una garantía ampliada, en tanto que el $50\%$ de los que compraron un modelo de lujo, también lo hicieron. Si se sabe que un comprador seleccionado al azar tiene una garantía ampliada, ¿qué tan probable es que él o ella tengan un modelo básico?\\
$A:\ Se\ vende\ un\ modelo\ basico$\\
$B:\ Garatia\ modelo\ basico$\\
$C:\ Garantia\ modelo\ de\ lujo$\\
$P(A|B)=\displaystyle\frac{P(A)P(B|A)}{P(B)}=\displaystyle\frac{(0.4)(0.3)}{0.42}=0.28$
%%%%Aqui empieza 64 por Gerardo Ayala Juarez%%%%%%%%%%%%%%%%%%%%%%%%%%%%%%%%%%%%%%%%%%%%%%%
\textbf{64.} \textit{Incidencia de una enfermedad rara}. Sólo 1 en 25 adultos padece una enfermedad rara para la cual se ha creado una prueba de diagnóstico. La prueba es tal que cuando un individuo que en realidad tiene la enfermedad, un resultado positivo se presentará en 99\%  de las veces mientras que en individuos sin enfermedad el examen será positivo sólo en un 2\% de las veces. %% Problema adaptado del ejemplo 2.30 donde se cambio 1 de 1000 a 1.25
\begin{itemize}
    \item[A$_1$]: Tiene la enfermedad - P: 0.04
    \item[A$_2$]: Tiene la enfermedad - P: 0.96
    \item[B]: El resultado es positivo - P: X
    \begin{itemize}
        \item P(B|A$_1$) = 0.99
        \item P(B|A$_2$) = 0.02
    \end{itemize}
    \item[a)] Dado que el resultado es positivo,¿Cuál es entonces la probabilidad de un resultado de prueba positivo?
    \item[R:] \vskip .05 cm
    P(A$_1 \cap $ B) = P($A_1$) $\cdot$ P(B|$A_1$) = $\displaystyle\frac{P(A_1)\cdot P(A_1\cap B)}{P(A_1)}$ =  0.04 $\cdot$ 0.99 = 0.0396
    \vskip .05 cm
    P(A$_2 \cap $ B) = P($A_2$) $\cdot$ P(B|$A_2$) = $\displaystyle\frac{P(A_2)\cdot P(A_2\cap B)}{P(A_2)}$ =  0.96 $\cdot$ 0.02 = 0.0192
    \vskip .05 cm
    P(B) = $P(A_1 \cap B) + P(A_2 \cap B) $ = 0.0588 = X
    \item[b)] Dado que el resultado de prueba es positivo, ¿cuál es la probabilidad de que el individuo tenga la enfermedad?
    \item[R:] P($A_1$|B) = $\displaystyle\frac{P(A_1\cap B)}{P(B)}$ = $\displaystyle\frac{0.0396}{0.0588}$ = 0.673
    \item[c)] Dado un resultado de prueba negativo, ¿cuál es la probabilidad de que el individuo no tenga la enfermedad?
    \item[R:] P($A_2$|B) = $\displaystyle\frac{P(A_2\cap B)}{P(B)}$ = $\displaystyle\frac{0.0192}{0.0588}$ = 0.326
\end{itemize}

\textbf{66.} Considere la siguiente información sobre vacacionistas (basada en parte en una encuesta reciente de Travelocity): 40\% revisan su correo electrónico de trabajo, 30\% utilizan un teléfono celular para permanecer en contacto con su trabajo, 25\% trajeron una computadora portátil consigo, 23\% revisan su correo electrónico de trabajo y utilizan un teléfono celular para permanecer en contacto y 51\% ni revisan su correo electrónico de trabajo ni utilizan un teléfono celular para permanecer en contacto ni trajeron consigo una computadora portátil. Además, 88 de cada 100 que traen una computadora portátil también revisan su correo electrónico de trabajo y 70 de cada 100 que utilizan un teléfono celular para permanecer en contacto también traen una computadora portátil.

\begin{itemize}
    \item[A:] Revisan correo electrónico - P: 0.40
    \item[B:] Utilizan tel\'efono - P: 0.30
    \item[C:] Computadora portatil - P: 0.25
    \item[A$\cap$B] P: 0.23
    \item[P(A|C)] P: 88/100
    \item[P(C|B)] P: 70/100
    \item[P(A$\cap$C)] = P($C$) $\cdot$ P(A|C) = $\displaystyle\frac{P(C)\cdot P(A \cap C)}{P(C)}$ =  0.11 $\cdot$ .88 = 0.22
    \item[P(C$\cap$B)] = P($B$) $\cdot$ P(C|B) = $\displaystyle\frac{P(B)\cdot P(C \cap B)}{P(B)}$ =  0.30 $\cdot$ .70 = 0.21
    \item[a)] ¿Cuál es la probabilidad de que un vacacionista seleccionado al azar que revisa su correo electrónico de trabajo también utilice un teléfono celular para permanecer en contacto?
    \item[R:]  P(B|A) = $\displaystyle\frac{P(A\cap B)}{P(A)}$ = $\displaystyle\frac{0.23}{0.40}$ = 0.575
    \item[b)] ¿Cuál es la probabilidad de que alguien que trae una computadora portátil también utilice un teléfono celular para permanecer en contacto?
    \item[R:] P(B|C) = $\displaystyle\frac{P(B\cap C)}{P(C)}$ = $\displaystyle\frac{.21}{.25}$ = 0.84
    \item[c)] Si el vacacionista seleccionado al azar revisó su correo electrónico de trabajo y trajo una computadora portátil, ¿cuál es la probabilidad de que él o ella utilice un teléfono celular para permanecer en contacto?
\end{itemize}

\textbf{68.} Una amiga que vive en Los Ángeles hace viajes frecuentes de consultoría a Washington, D.C.; 50\% del tiempo viaja en la línea aérea \#1, 30\% del tiempo en la aerolínea \#2 y el 20\% restante en la aerolínea \#3. Los vuelos de la aerolínea \#1 llegan demorados a D.C. 30\% del tiempo y 10\% del tiempo llegan demorados a L.A. Para la aerolínea \#2, estos porcentajes son 25\% y 20\%, en tanto que para la aerolínea \#3 los porcentajes son 40\% y 25\%. Si se sabe que en un viaje particular ella llegó demorada a exactamente uno de los destinos, ¿cuáles son las probabilidades posteriores de haber volado en las aerolíneas \#1, \#2 y \#3? Suponga que la probabilidad de arribar con demora a L.A. no se ve afectada por lo que suceda en el vuelo a D.C. [Sugerencia: Desde la punta de cada rama de primera generación en un diagrama de árbol, trace tres ramas de segunda generación identificadas, respectivamente, como, 0 demorado, 1 demorado y 2 demorado.]
\begin{itemize}
    \item [$A_1$] : Viaja Aerol\'inea \# 1 - P: 50\%
    \item [$A_2$] : Viaja Aerol\'inea \# 2 - P: 30\%
    \item [$A_3$] : Viaja Aerol\'inea \# 3 - P: 20\%
    \item [$B$] : Se retrasa el vuelo
    \item [$B_{1.1}$] : Se retrasa el vuelo D.C. Aerolina \#1 - P : 30\%
    \item [$B_{1.2}$] : Se retrasa el vuelo L.A. Aerolina \#1 - P : 10\%
    \item [$B_{2.1}$] : Se retrasa el vuelo D.C. Aerolina \#2 - P : 25\%
    \item [$B_{2.2}$] : Se retrasa el vuelo L.A. Aerolina \#2 - P : 20\%
    \item [$B_{3.1}$] : Se retrasa el vuelo D.C. Aerolina \#3 - P : 40\%
    \item [$B_{3.2}$] : Se retrasa el vuelo L.A. Aerolina \#3 - P : 25\%
    \item [P(B)] : $P(A_1 \cap (B_{1.1} \cap B_{1.2}^{c}))$ + $P(A_1 \cap (B_{1.2} \cap B_{1.1}^{c}))$ + $P(A_1 \cap (B_{1.1} \cap B_{1.2}))$ + $P(A_2 \cap (B_{2.1} \cap B_{2.2}^{c})) + P(A_2 \cap (B_{2.2} \cap B_{2.1}^{c}))$ + $P(A_2 \cap (B_{2.1} \cap B_{2.2}))$ + $P(A_3 \cap (B_{3.1} \cap B_{3.2}^{c}))$ + $P(A_3 \cap (B_{3.2} \cap B_{3.1}^{c}))$ + $P(A_3 \cap (B_{3.1} \cap B_{3.2}))$
    \item Recordando: \textit{Suponga que la probabilidad de arribar con demora a L.A. no se ve afectada por lo que suceda en el vuelo a D.C.} Podemos suponer que son eventos independientes por lo tanto llegamos a lo siguiente.
    \item [P(B)] : (0.50)(0.30)(0.90) + (0.50)(0.10)(0.70) + (0.50)(0.30)(0.10) + (0.30)(0.25)(0.80) + (0.30)(0.20)(0.75) + (0.30)(0.20)(0.25) + (0.20)(0.40)(0.75) + (0.20)(0.25)(0.60) + (0.20)(0.25)(0.40)
    \item [P(B)] : Se retrasa el vuelo - P: .415
    \item [a)]  Si se sabe que en un viaje particular ella llegó demorada a exactamente uno de los destinos, ¿cuáles son las probabilidades posteriores de haber volado en las aerolíneas \\#1, \\#2 y \\#3?
    \begin{itemize}
        \item $\displaystyle\frac{P(A_1 \cap (B_{1.1} \cap B_{1.2}^{c}))}{P(B)}$ = $\displaystyle\frac{0.135}{0.415}$ = 0.3253
        \item $\displaystyle\frac{P(A_1 \cap (B_{1.2} \cap B_{1.1}^{c}))}{P(B)}$ = $\displaystyle\frac{0.035}{0.415}$ = 0.0843
    \end{itemize}
    \item[Probabilidad] de haber volado en la aerol\'inea \\#1 0.3253 + 0.0843 = 0.4096
    \begin{itemize}
        \item $\displaystyle\frac{P(A_2 \cap (B_{2.1} \cap B_{2.2}^{c}))}{P(B)}$ = $\displaystyle\frac{0.060}{0.415}$ = 0.1445
        \item $\displaystyle\frac{P(A_2 \cap (B_{2.2} \cap B_{2.1}^{c}))}{P(B)}$ = $\displaystyle\frac{0.045}{0.415}$ = 0.1084
    \end{itemize}
    \item[Probabilidad] de haber volado en la aerol\'inea \\#2 0.1445 + 0.1084 = 0.2529
    \begin{itemize}
        \item $\displaystyle\frac{P(A_3 \cap (B_{3.1} \cap B_{3.2}^{c}))}{P(B)}$ = $\displaystyle\frac{0.060}{0.415}$ = 0.1445
        \item $\displaystyle\frac{P(A_3 \cap (B_{3.2} \cap B_{3.1}^{c}))}{P(B)}$ = $\displaystyle\frac{0.030}{0.415}$ = 0.0722
    \end{itemize}
    \item[Probabilidad] de haber volado en la aerol\'inea \\#3 0.1145 + 0.0722 = 0.2167
\end{itemize}
\end{document}
