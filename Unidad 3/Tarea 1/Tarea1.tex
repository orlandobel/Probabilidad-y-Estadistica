\documentclass[12pt, letterpaper]{article}
\usepackage{babel}
\usepackage[T1]{fontenc}
\usepackage{textcomp}
\usepackage[utf8]{inputenc} % Puede depender del sistema o editor
\usepackage{amsmath}
\usepackage{amsfonts}
\usepackage{amssymb}
\usepackage[left=2cm,right=2cm,top=2cm,bottom=2cm]{geometry}
\usepackage{mathrsfs}
\usepackage{pstricks-add}
\usepackage{pgfkeys}
\usepackage{setspace}
\usepackage{fancyhdr}
\usepackage{graphicx}
\usepackage{enumerate}

\usepackage{tikz}
\usetikzlibrary{fit,positioning}
\begin{document}
    \begin{titlepage}
        \centering
        {\scshape\LARGE Instituto Politécnico Nacional\\ Unidad Profesional Interdisiplinaria de Ingenierias campus Zacatecas\par}
        \vspace{1cm}
        {\scshape\Large Probabilidad Y Estadistica\par}
        \vspace{1.5cm}
        {\huge\bfseries Unidad 3 Tarea 1\par}
        \vspace{2cm}
        {\Large\itshape Gerardo Ayala Juárez\par}
        {\Large\itshape Olando Odiseo Belmonte Flores\par}
        {\Large\itshape Lucía Monserrat López Méndez\par}
        {\Large\itshape Oscar Iván Palacios Ulloa\par}
        \vfill
        Maestro:\par
        \textsc{
        Rosendo Vasquez Bañuelos}
        \vfill
        % Bottom of the page
        {\large \today \par}
    \end{titlepage}

    \textbf{2. }\vskip1cm
    \textbf{4. }\vskip1cm
    \textbf{6. }\vskip1cm
    \textbf{8. }De acuerdo con un estudio publicado por un grupo de sociólogos de la Universidad de Massachusetts,
    aproximadamente el $60\%$ de los adictos al Valium en el estado de Massachusetts, lo tomaon por primera vez debido
    a problemas sicológicos. Encuentre la probabilidad de que de los siguientes 8 adictos entrevistados
    \begin{enumerate}[a)]
        \item Exactamente 3 hayan comenzado a usarlo debido a problemas sicológicos\\
            $b(3;8,0.6)=
            \left(\begin{array}{c l}
                      8\\3
            \end{array}\right)
            (0.6)^{3}(0.4)^{5}=0.12$
        \item Al menos 5 de ellos comenzara a tomarlo por problemas que no fueron sicológicos\\
            $x$ representa el numero de adictos que comenzaron a tomarlo por problemas no sicológicos
            $p(x<5)=\displaystyle\sum_{x=0}^{4}b(x;8,0.4)=0.8263$
    \end{enumerate}\vskip1cm

    \textbf{10. }De acuerdo con un reporte publicado en la revista \textit{Padre}, Septiemre 14 de 1980, una investigación
    a nivel nacional llevada a cabo por la Universidad de Michigan reveló que casi el $70\%$ de los estudiantes del
    último año desaprueban las medidas para controlar el hábito de fumar mariguana todos los días. Si 12 de estos
    estudiantes se seleccionan al azar y se les pregunta su opinión, encuentre la probabilidad de que el número que
    desaprueba dicha medida sea
    \begin{enumerate}
        \item Cualquier camtidad entre 7 y 9
            $P(7\leq x\leq 9)=\displaystyle\sum_{x=0}^{9}b(x;12,0.7)-\sum_{x=0}^{6b(x;12,0.7)}=0.7472-0.1178=0.6294$
        \item Cuando mucho 5
            $P(x\leq 5)=\displaystyle\sum_{x=0}^{5}b(x;12,0.7)=0.0386$
        \item no menos de 8
            $P(x\geq 8)=\displaystyle\sum_{x=0}^{12}b(x;12,0.7)-\sum_{x=0}^{7}b(x;12,0.7)=1-0.2763=0.7237$
    \end{enumerate}\vskip1cm

    \textbf{12. }\vskip1cm
    \textbf{14. }\vskip1cm
    \textbf{16.} Suponga que los motores de un aeroplano operan en forma independiente y de que fallan con una probabilidad de 0.4. Suponiendo que  uno de estos artefactors realiza un vuelo seguro en tanto se mantenga funcionando cuando menos la mitad  de los motores , determine qu\'e aeroplano uno de los 4 motores o uno de 2, tiene mayor probabilidad de terminar su vuelo exitosamente.\\

        \begin{center}
            \begin{math}
                \begin{array}{cc||cc}
                n=2 ;&p=0.4,q=0.6&n=4 ;&p=0.4,q=0.6\\
                    P(x\leq 1)=&\displaystyle\sum^{1}_{i=0} b(i;2,0.4)&P(x\leq 2)=&\displaystyle\sum^{2}_{j=0} b(h;4,0.4)\\ 
                    \displaystyle\sum^{1}_{i=0} b(i;2,0.4)=&.84&\displaystyle\sum^{2}_{j=0} b(j;4,0.4)=&.8208\\    
                \end{array}
         \end{math}
        \\Por lo tanto, tiene mas probabilidades de Tener un vuelo exitoso un vuelo con dos motores que 4 motores 
        \end{center}\vskip1cm
        \textbf{18.} Encuentre la media y la varianca de la variable aleatoria binomal del ejercicio 14.\\
        $b(x;n,p)$\\
        $n=20$\\
        $p=0.20$\\
        $q=1-p=0.80$\\
        \begin{center}
            \begin{math}
                \begin{array}{|c c || c c|}
                    \hline
                    \mu =& np & \varphi^2 =& npq \\
                    \mu =& (20)(0.20) & \varphi^2 =&(20)(0.20)(0.80)\\
                    \hline
                    \hline
                    \mu =&  4 & \varphi^2 =& 3.2 \\    
                    \hline
                \end{array} 
            \end{math}
        \end{center}\vskip1cm
    \textbf{20. }\vskip1cm
    \textbf{22. }\vskip1cm
\end{document}
