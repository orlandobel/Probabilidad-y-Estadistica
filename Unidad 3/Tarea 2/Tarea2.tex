\documentclass[12pt, letterpaper]{article}
\usepackage{babel}
\usepackage[T1]{fontenc}
\usepackage{textcomp}
\usepackage[utf8]{inputenc} % Puede depender del sistema o editor
\usepackage{amsmath}
\usepackage{amsfonts}
\usepackage{amssymb}
\usepackage[left=2cm,right=2cm,top=2cm,bottom=2cm]{geometry}
\usepackage{mathrsfs}
\usepackage{pstricks-add}
\usepackage{pgfkeys}
\usepackage{setspace}
\usepackage{fancyhdr}
\usepackage{graphicx}
\usepackage{enumerate}

\usepackage{tikz}
\usetikzlibrary{fit,positioning}
\begin{document}
    \begin{titlepage}
        \centering
        {\scshape\LARGE Instituto Politécnico Nacional\\ Unidad Profesional Interdisiplinaria de Ingenierias campus Zacatecas\par}
        \vspace{1cm}
        {\scshape\Large Probabilidad Y Estadistica\par}
        \vspace{1.5cm}
        {\huge\bfseries Unidad 3 Tarea 2\par}
        \vspace{2cm}
        {\Large\itshape Gerardo Ayala Juárez\par}
        {\Large\itshape Olando Odiseo Belmonte Flores\par}
        {\Large\itshape Lucía Monserrat López Méndez\par}
        {\Large\itshape Oscar Iván Palacios Ulloa\par}
        \vfill
        Maestro:\par
        \textsc{
        Rosendo Vasquez Bañuelos}
        \vfill
        % Bottom of the page
        {\large \today \par}
    \end{titlepage}
    \textbf{2. }\vskip1cm
    \textbf{4. }De un lote de 10 proyectiles, 4 se seleccionan al azar y se disparan. Si el lote contien 3 proectiles
    defectuosos que no explotan, ¿Cuál es la probabilidad de que
    \begin{enumerate}[a)]
        \item los 4 exploten?\\
            $b(4;4,0.3)=
            \left(\begin{array}{c l}
                            4\\4
            \end{array}\right)
            (0.3)^4(0.7)^0=0.0081$
        \item al menos 2 no exploten?\\
            $P(x\geq 2)=\displaystyle\sum_{x=0}^{4}b(x;4,0.3)-\sum_{x=0}^{1}b(x;4,0.3)=1-0.9163=0.0837$
    \end{enumerate}\vskip1cm

    \textbf{6. }¿Cuál es la probailidad de que una mesera se rehúse a servir bebidas alcohólocas únicamente a 2 menores
    de edad, si verifica aleatoriamente sólo 5 identificaciones de entre 9 estudiantes, de los cuales 4 no tienen la
    edad suficiente\vskip0.3cm
        $b(2;5,\frac{4}{9})=
        \left(\begin{array}{c l}
                  5\\2
        \end{array}\right)
        \displaystyle\left(\frac{4}{9}\right)^2\left(\frac{5}{9}\right)^3=0.1045$\vskip1cm

    \textbf{8. }\vskip1cm
    \textbf{10. }\vskip1cm
    \textbf{12. }\vskip1cm
    \textbf{14. }\vskip1cm
    \textbf{16. }\vskip1cm
    \textbf{18. }\vskip1cm
\end{document}