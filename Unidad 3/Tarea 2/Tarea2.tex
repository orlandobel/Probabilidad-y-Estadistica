\documentclass[12pt, letterpaper]{article}
\usepackage{babel}
\usepackage[T1]{fontenc}
\usepackage{textcomp}
\usepackage[utf8]{inputenc} % Puede depender del sistema o editor
\usepackage{amsmath}
\usepackage{amsfonts}
\usepackage{amssymb}
\usepackage[left=2cm,right=2cm,top=2cm,bottom=2cm]{geometry}
\usepackage{mathrsfs}
\usepackage{pstricks-add}
\usepackage{pgfkeys}
\usepackage{setspace}
\usepackage{fancyhdr}
\usepackage{graphicx}
\usepackage{enumerate}

\usepackage{tikz}
\usetikzlibrary{fit,positioning}
\begin{document}
    \begin{titlepage}
        \centering
        {\scshape\LARGE Instituto Politécnico Nacional\\ Unidad Profesional Interdisiplinaria de Ingenierias campus Zacatecas\par}
        \vspace{1cm}
        {\scshape\Large Probabilidad Y Estadistica\par}
        \vspace{1.5cm}
        {\huge\bfseries Unidad 3 Tarea 2\par}
        \vspace{2cm}
        {\Large\itshape Gerardo Ayala Juárez\par}
        {\Large\itshape Olando Odiseo Belmonte Flores\par}
        {\Large\itshape Lucía Monserrat López Méndez\par}
        {\Large\itshape Oscar Iván Palacios Ulloa\par}
        \vfill
        Maestro:\par
        \textsc{
        Rosendo Vasquez Bañuelos}
        \vfill
        % Bottom of the page
        {\large \today \par}
    \end{titlepage}
    \textbf{12.} Se estima que 4000 de los 10 000 residentes que votan en un pueblo est\'an en contra del nuevo impuesto sobre las ventas. Si se seleccionan aleatoriamente 15 votantes y se les pregunta su opini\'on, ¿Cu\'al es la probabilidad de que almenos 7 est\'en a favor del nuevo impuesto?\\
    $N=10000$\\
    $k=4000$\\
    $n=15$\\
    $7\leq x \leq 15$\\
    \begin{center}
        \begin{math}
            \begin{array}{ccc}
                h(x;N,n,k)=&\displaystyle\frac{{\displaystyle\binom{k}{x} \displaystyle\binom{N-k}{n-x}}}{\displaystyle\binom{N}{n}}&\\
                P(7\leq x)=&\displaystyle\sum^{15}_{i=7} h(i;10000,15,4000)&\\ 
                h(7;10000,15,4000)= & \frac{\binom{4000}{7} \binom{6000}{8}}{\binom{10000}{15}}&=0.17718\\
                h(8;10000,15,4000)= & \frac{\binom{4000}{8} \binom{6000}{7}}{\binom{10000}{15}}&=0.11805\\
                h(9;10000,15,4000)= & \frac{\binom{4000}{9} \binom{6000}{6}}{\binom{10000}{15}}&=0.06115\\
                h(10;10000,15,4000)= & \frac{\binom{4000}{10} \binom{6000}{5}}{\binom{10000}{15}}&=0.02442\\
                h(11;10000,15,4000)= & \frac{\binom{4000}{11} \binom{6000}{4}}{\binom{10000}{15}}&=7.3x10^{-3}\\
                h(12;10000,15,4000)= & \frac{\binom{4000}{12} \binom{6000}{3}}{\binom{10000}{15}}&=1.6x10^{-3}\\
                h(13;10000,15,4000)= & \frac{\binom{4000}{13} \binom{6000}{2}}{\binom{10000}{15}}&=2.5x10^{-4}\\
                h(14;10000,15,4000)= & \frac{\binom{4000}{14} \binom{6000}{1}}{\binom{10000}{15}}&=2.3x10^{-5}\\
                h(15;10000,15,4000)= & \frac{\binom{4000}{15} \binom{6000}{0}}{\binom{10000}{15}}&=1.05x10^{-6}\\
                P(7\leq x)=&\displaystyle\sum^{15}_{i=7} h(i;10000,15,4000)&=0.38997\\
                \hline
                \end{array}
        \end{math}\\
        Por otro lado si: $N>>n \hspace{1 cm}   p = \displaystyle\frac{k}{N} :  h(x;N,n,k) \approx b(x;n,p)$\\
        \begin{math}
            \begin{array}{c c c}
                P(7\leq x)=&\displaystyle\sum^{15}_{i=0} b(i;15,\displaystyle\frac{4}{10})- \displaystyle\sum^{6}_{i=0} b(i;15,\displaystyle\frac{4}{10})&\\
                P(7\leq x)=&1-0.6098&= 0.3902\\   
            \end{array}
        \end{math}
    \end{center}
    \textbf{14.} De entre 150 solicitudes para empearse en la IRS en una gran ciudad, s\'olo 30 son de mujeres. Si 10 de los solicitantes se escogenal azar para dar asistencia libre sobre impuestos a los residentes de esta ciudad, utilice la aproximaci\'on binomial a la distribuci\'ion hipergeom\'etrica para encontrar la probabilidad de que al menos 3 mujeres sean seleccionadas.\\
    N=150\\
    k=30\\
    n=10\\
    \begin{center}
        \begin{math}
            \begin{array}{ccc}
                \displaystyle\frac{k}{N} :&  h(x;N,n,k) \approx b(x;n,p)&\\
                \displaystyle\frac{30}{150} :&  h(x;150,10,30) \approx b(x;10,\displaystyle\frac{30}{150})&\\
                \hline\\
                P(3\leq x)=&\displaystyle\sum^{10}_{i=0} b(i;10,\displaystyle\frac{3}{15})- \displaystyle\sum^{2}_{i=0} b(i;10,\displaystyle\frac{3}{15})&\\
                P(3 \leq x)=& 1-0.6778&=0.3222\\
            \end{array}
        \end{math}
    \end{center}\vskip1cm
    Comprobando con la hipergeometrica:\\
    \begin{center}
        \begin{math}
            \begin{array}{ccc}
                h(x;N,n,k)=&\displaystyle\frac{{\displaystyle\binom{k}{x} \displaystyle\binom{N-k}{n-x}}}{\displaystyle\binom{N}{n}}&\\
                P(3\leq x)=&\displaystyle\sum^{10}_{i=3} h(i;150,10,30)&\\ 
                h(3;150,10,30)= & \frac{\binom{30}{3} \binom{120}{7}}{\binom{150}{10}}&=0.2065\\
                h(4;150,10,30)= & \frac{\binom{30}{4} \binom{120}{6}}{\binom{150}{10}}&=0.0855\\
                h(5;150,10,30)= & \frac{\binom{30}{5} \binom{120}{5}}{\binom{150}{10}}&=0.0232\\
                h(6;150,10,30)= & \frac{\binom{30}{6} \binom{120}{4}}{\binom{150}{10}}&=4.1x10^{-3}\\
                h(7;150,10,30)= & \frac{\binom{30}{7} \binom{120}{3}}{\binom{150}{10}}&=4.8x10^{-4}\\
                h(8;150,10,30)= & \frac{\binom{30}{8} \binom{120}{2}}{\binom{150}{10}}&=3.5x10^{-5}\\
                h(9;150,10,30)= & \frac{\binom{30}{9} \binom{120}{1}}{\binom{150}{10}}&=1.46x10^{-6}\\
                h(10;150,10,30)= & \frac{\binom{30}{10} \binom{120}{0}}{\binom{10000}{15}}&=2.56x10^{-8}\\
                P(3\leq x)=&\displaystyle\sum^{10}_{i=3} h(i;150,10,30)&=0.3198\\
                \hline
                \end{array}
        \end{math}\\
    \end{center}\vskip1cm 
\end{document}