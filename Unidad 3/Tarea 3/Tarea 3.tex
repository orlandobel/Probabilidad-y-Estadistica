\documentclass[12pt, letterpaper]{article}
\usepackage{babel}
\usepackage[T1]{fontenc}
\usepackage{textcomp}
\usepackage[utf8]{inputenc} % Puede depender del sistema o editor
\usepackage{amsmath}
\usepackage{amsfonts}
\usepackage{amssymb}
\usepackage[left=2cm,right=2cm,top=2cm,bottom=2cm]{geometry}
\usepackage{mathrsfs}
\usepackage{pstricks-add}
\usepackage{pgfkeys}
\usepackage{setspace}
\usepackage{fancyhdr}
\usepackage{graphicx}
\usepackage{enumerate}

\usepackage{tikz}
\usetikzlibrary{fit,positioning}
\begin{document}
    \begin{titlepage}
        \centering
        {\scshape\LARGE Instituto Politécnico Nacional\\ Unidad Profesional Interdisiplinaria de Ingenierias campus Zacatecas\par}
        \vspace{1cm}
        {\scshape\Large Probabilidad Y Estadistica\par}
        \vspace{1.5cm}
        {\huge\bfseries Unidad 3 Tarea 3\par}
        \vspace{2cm}
        {\Large\itshape Gerardo Ayala Juárez\par}
        {\Large\itshape Olando Odiseo Belmonte Flores\par}
        {\Large\itshape Lucía Monserrat López Méndez\par}
        {\Large\itshape Oscar Iván Palacios Ulloa\par}
        \vfill
        Maestro:\par
        \textsc{
        Rosendo Vasquez Bañuelos}
        \vfill
        % Bottom of the page
        {\large \today \par}
    \end{titlepage}

    \textbf{2. }Un científico inocula varios ratones , uno a la vez, con u germen de una enfermedad hasta que obtiene 2
    que la han contraído. Si la probabilidad  de contraer la enfermedad es $1/6$, ¿Cuál es la probabilidad de que se
    requieran 8 ratones?\vskip1cm

    \textbf{4. }Encuentre la probabilidad  de que una persona que lanza una monda obtenga\\
    \begin{enumerate}[a)]
        \item la terecera cara en el séptimo lanzamiento
        \item la primera cara en el cuarto lanzamiento
    \end{enumerate}\vskip1cm

    \textbf{6. }\vskip1cm

    \textbf{8. }\vskip1cm

    \textbf{10. }\vskip1cm

    \textbf{12. }\vskip1cm

    \textbf{14. }\vskip1cm

    \textbf{16. }\vskip1cm

    \textbf{18. }\vskip1cm



\end{document}