\documentclass[12pt, letterpaper]{article}
\usepackage{babel}
\usepackage[T1]{fontenc}
\usepackage{textcomp}
\usepackage[utf8]{inputenc} % Puede depender del sistema o editor
\usepackage{amsmath}
\usepackage{amsfonts}
\usepackage{amssymb}
\usepackage[left=2cm,right=2cm,top=2cm,bottom=2cm]{geometry}
\usepackage{mathrsfs}
\usepackage{pstricks-add}
\usepackage{pgfkeys}
\usepackage{setspace}
\usepackage{fancyhdr}
\usepackage{graphicx}
\usepackage{enumerate}

\usepackage{tikz}
\usetikzlibrary{fit,positioning}
\begin{document}
    \begin{titlepage}
        \centering
        {\scshape\LARGE Instituto Politécnico Nacional\\ Unidad Profesional Interdisiplinaria de Ingenierias campus Zacatecas\par}
        \vspace{1cm}
        {\scshape\Large Probabilidad Y Estadistica\par}
        \vspace{1.5cm}
        {\huge\bfseries Unidad 3 Tarea 3\par}
        \vspace{2cm}
        {\Large\itshape Gerardo Ayala Juárez\par}
        {\Large\itshape Olando Odiseo Belmonte Flores\par}
        {\Large\itshape Lucía Monserrat López Méndez\par}
        {\Large\itshape Oscar Iván Palacios Ulloa\par}
        \vfill
        Maestro:\par
        \textsc{
        Rosendo Vasquez Bañuelos}
        \vfill
        % Bottom of the page
        {\large \today \par}
    \end{titlepage}

    \textbf{2. }Un científico inocula varios ratones , uno a la vez, con u germen de una enfermedad hasta que obtiene 2
    que la han contraído. Si la probabilidad  de contraer la enfermedad es $1/6$, ¿Cuál es la probabilidad de que se
    requieran 8 ratones?\vskip1cm

    \textbf{4. }Encuentre la probabilidad  de que una persona que lanza una monda obtenga\\
    \begin{enumerate}[a)]
        \item la terecera cara en el séptimo lanzamiento
        \item la primera cara en el cuarto lanzamiento
    \end{enumerate}\vskip1cm

    \textbf{6. } De acuerdo con un estudio publicado por un grupo de sociólogos de la Universidad de Massachusetts, alrededor de las dos terceras partes de los 20 millones de personas en Estados Unidos que consumen Valium son mujeres. Suponiendo que ésta es una estimación válida, encuentre la probabilidad de que en un determinado día la quinta receta médica por Valium sea\\
$b^*(x;k,P)=\left( \frac{x-1}{k-1}\right) P^k q^{x-k}$\\    
    \begin{enumerate}[a)]
    	\item La primera preinscripción de Valium para una mujer.\\
    	$x=5$\\
    	$k=1$\\
    	$P=\frac{2}{3}$\\
    	$q=1-\frac{2}{3}$\\
    	$b^*(5;1,\frac{2}{3})=\left( \frac{4}{0}\right) \left( \frac{2}{3}\right)^1 \left( 1-\frac{2}{3}\right)^{4}$\\
    	$b^*(5;1,\frac{2}{3})=\left( \frac{2}{3}\right)^k \left( 1-\frac{2}{3}\right)^{4}$\\
    	$b^*(5;1,\frac{2}{3})=\frac{2}{3}\left(\frac{1}{8}\right)=0.0082$//
    	\item La tercera preinscripción de Valium para una mujer.\\
    	$x=5$\\
    	$k=3$\\
    	$P=\frac{2}{3}$\\
    	$q=1-\frac{2}{3}$\\
    	$b^*(5;3,\frac{2}{3})=\left( \frac{4}{2}\right) \left( \frac{2}{3}\right)^3 \left( 1-\frac{2}{3}\right)^{2}$\\
    	$b^*(5;3,\frac{2}{3})=2\left( \frac{8}{27}\right) \left( \frac{1}{9}\right)= \frac{16}{243}=0.0658$\\
    	
    	
    \end{enumerate}
    
    \vskip1cm

    \textbf{8. }En promedio, en una cierta intersección ocurren 3 accidentes víales por mes. ¿Cúal es la probabilidad de que en un determinado mes en esta intersección\\
	$\mu =3$\\    
    \begin{enumerate}[a)]
    \item Ocurran exactamente 5 accidentes?\\
	$x=5$\\    
    $P(x;\mu)= \frac{e^{-\mu}\mu ^x}{x!}$\\
    $P(5;3)= \frac{e^{-3}(3)^x}{5!}=0.1008$\\
    \item Ocurran menos de 3 accidentes?\\
    $x<3$
    $\sum^1_{x=0} P(x;3)=.1991$
    
    \end{enumerate}
    
    \vskip1cm

    \textbf{10. }\vskip1cm

    \textbf{12. }\vskip1cm

    \textbf{14. }\vskip1cm

    \textbf{16. }\vskip1cm

    \textbf{18. }\vskip1cm



\end{document}