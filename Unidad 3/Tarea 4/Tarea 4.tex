\documentclass[12pt, letterpaper]{article}
\usepackage{babel}
\usepackage[T1]{fontenc}
\usepackage{textcomp}
\usepackage[utf8]{inputenc} % Puede depender del sistema o editor
\usepackage{amsmath}
\usepackage{amsfonts}
\usepackage{amssymb}
\usepackage[left=2cm,right=2cm,top=2cm,bottom=2cm]{geometry}
\usepackage{mathrsfs}
\usepackage{pstricks-add}
\usepackage{pgfkeys}
\usepackage{setspace}
\usepackage{fancyhdr}
\usepackage{graphicx}
\usepackage{enumerate}

\usepackage{tikz}
\usetikzlibrary{fit,positioning}
\begin{document}
    \begin{titlepage}
        \centering
        {\scshape\LARGE Instituto Politécnico Nacional\\ Unidad Profesional Interdisiplinaria de Ingenierias campus Zacatecas\par}
        \vspace{1cm}
        {\scshape\Large Probabilidad Y Estadistica\par}
        \vspace{1.5cm}
        {\huge\bfseries Unidad 3 Tarea 4\par}
        \vspace{2cm}
        {\Large\itshape Gerardo Ayala Juárez\par}
        {\Large\itshape Olando Odiseo Belmonte Flores\par}
        {\Large\itshape Lucía Monserrat López Méndez\par}
        {\Large\itshape Oscar Iván Palacios Ulloa\par}
        {\Large\itshape José Mauricio Juanes Martínez\par}
        \vfill
        Maestro:\par
        \textsc{
        Rosendo Vasquez Bañuelos}
        \vfill
        % Bottom of the page
        {\large \today \par}
    \end{titlepage}

    \textbf{2. }Encuentre el valor de $z$ si el área bajo la curva normal estándar
        \begin{enumerate}[a)]
            \item a la derecha de $z$ es $1.43$\vskip0.5cm
                $z=1.07$
            \item a la izquierda de $z$ es $0.1131$\vskip0.5cm
                $z=-1.21$
            \item entre $0$ y $z$, con $z\geq 0$, es $0.4838$\vskip0.5cm
                $P(z\leq Z)=0.4838+P(Z\leq 0)=0.4838+0.5=0.9838$\vskip0.5cm
                $z=2.24$
            \item entre $-z$ y $z$, con $z\geq 0$, es $0.9500$\vskip0.5cm
                $P(0\leq z\leq Z)=\displaystyle\frac{0.9500}{2}=0.475$\vskip0.5cm
                $P(z\leq Z)=0.475+P(Z\leq 0)=0.475+0.5=0.975$\vskip0.5cm
                $z=1.96$
        \end{enumerate} \vskip1cm

    \textbf{4. } dada una distribución normal con $\mu = 30$ y $\sigma = 6$, encuentre
        \begin{enumerate}[a)]
            \item el área de la curva normal a la derecha de $x=17$\vskip0.5cm
                $z=\displaystyle\frac{17-30}{6}=-2.16$\vskip0.5cm
                $P(z\geq -2.16)=P(z\leq 2.16)=0.9846$
            \item el área de la curva normal a la izquierda de $x=22$\vskip0.5cm
                $z=\displaystyle\frac{22-30}{6}=-1.33$\vskip0.5cm
                $P(z\leq -1.33)=0.0918$
            \item el área de la curva normal entre $x=32$ y $x=41$\vskip0.5cm
                $z_1=\displaystyle\frac{32-30}{6}=0.33$\vskip0.5cm
                $z_2=\displaystyle\frac{41-30}{6}=1.83$\vskip0.5cm
                $P(0.33\leq z\leq 1.83)=P(z\leq 1.83)-P(z\leq 0.33)=0.9664-0.6293=0.3371$
            \item el valor de $x$ que tiene el $80\%$ del área de la curva normal a la izquierda\vskip0.5cm
                $z\simeq 0.48$\vskip0.5cm
                $x=0.48*6+30=32.88$
            \item los valres de $x$ que contienen un intervalo central de $75\%$ de la mitad del área de la curva normal

        \end{enumerate}\vskip1cm

    \textbf{6. }De acuerdo con el teorema de Chebyshev, la probabilidad de que cualquier variable aleatoria asuma un valor
    dentro de 3 desviaciones estándar de la media es al menos $8/9$. Si se sabe que la distribución de probabilidad de
    una variable aleatoria $X$ es normal con media $\mu$ y varianza $\sigma ^2$, ¿cuál es el valor exacto de
    $P(\mu - 3\sigma \leq X\leq \mu + 3\sigma )$?\vskip0.5cm
        $z_1=\displaystyle\frac{(\mu -3\sigma)-\mu}{\sigma}=\frac{-3\sigma}{\sigma}=-3$\vskip0.5cm
        $z_2=\displaystyle\frac{(\mu +3\sigma)-\mu}{\sigma}=\frac{3\sigma}{\sigma}=3$\vskip0.5cm
        $P(-3\leq z\leq 3)=P(z\leq 3)-P(z\leq -3)=0.9987-0.0013=0.9974$\vskip0.5cm
        $P(\mu - 3\sigma \leq X\leq \mu + 3\sigma )=0.9974$
    \vskip1cm


    \textbf{8. }Las piezas de pan del centro distribuidas a las tiendas locales por una cierta pasteleria tienen una
    longitud promedio de $30cm$ y una desviación estándar de $2cm$. Suponiendo que las longiudes  están normalmente
    distribuidas, ¿qué porcentaje de las piezas son
        \begin{enumerate}[a)]
            \item de mas de $31.7$ centímetros de longitud?\vskip0.5cm
                $z=\displaystyle\frac{31.7-30}{2} = 0.85$\vskip0.5cm
                $p(z\geq 0.85)=p(z\leq -0.85)=0.1977$
            \item entre $29.3$ y $33.5$ centímetros de longitud?\vskip0.5cm
                $z_1=\displaystyle\frac{29.3-30}{2}=-0.35$\vskip0.5cm
                $z_2=\displaystyle\frac{33.5-30}{2}=1.75$\vskip0.5cm
                $p(-0.35\leq z\leq 1.75)=p(z\leq 1.75)-p(z\leq -0.35)=0.9599-0.3632=0.5967$
            \item de una longitud menor que $25.5$ centímetros?\vskip0.5cm
                $z=\displaystyle\frac{25.5-30}{2}=-2.25$\vskip0.5cm
                $p(z\leq -2.25)=0.0122$
        \end{enumerate}\vskip1cm

    \textbf{10. }El diámetro interno ya terminado de un anillo de pistón está normalmente distribuido con una media de $10$
    centímetros y una desviación estándar de $0.03$ centímetros.
        \begin{enumerate}[a)]
            \item ¿Qué proporcion de los anillos tendrá un diámetro interno que exceda de $10.075$ centímetros?\vskip0.5cm
                $z=\displaystyle\frac{10.075-20}{0.03}=2.5$\vskip0.5cm
                $p(z\geq 2.5)=p(z\leq -2.5)=0.0062$
            \item ¿Cuál es la probabilidad de que un anillo de pistón tenga un diámetro interno entre $9.97$ y $10.03$
                    centímetros?\vskip0.5cm
                $z_1=\displaystyle\frac{9.97-10}{0.03}=-1$\vskip0.5cm
                $z_2=\displaystyle\frac{10.03-10}{0.03}=1$\vskip0.5cm
                $p(-1\leq z\leq 1)=p(z\leq 1)-P(z\leq -1)=0.8413-0.1587=0.6826$
            \item ¿Abajo de qué valor de diámetro interno caerá el $15\%$ de los anillos de pistón?\vskip0.5cm
                $z\simeq -1.04$\vskip0.5cm
                $x=-1.04*0.03+10=9.9688\simeq 9.97$
        \end{enumerate}\vskip1cm

    \textbf{12. }Si un conjunto de calificaciónes de un examen de estadística se aproxima a una distribución normal con
    una media de $74$ y una desviación estándar de $7.9$, encuentre
        \begin{enumerate}[a)]
            \item la calificación más baja de pase si al $10\%$ de los estudiantes más bajos se le dio una NA (No Acreditado)\vskip0.5cm
                $z\simeq -1.28$\vskip0.5cm
                $x=-1.28*7.9+74=63.888\simeq 63.89$\vskip0.5cm
            \item la B (Bien) más alta si al $5\%$ superior de los estudiantes se le dio MB (Muy Bien)\vskip0.5cm
                $z\simeq 1.65$\vskip0.5cm
                $x=1.65*7.9+74=87.035\simeq 87$
            \item La B más baja si al $10\%$ superior de los estudiantes se le dio MB y al siguiente $25\%$ se le dio B\vskip0.5cm
                $p(z\leq -z_2)=0.25+P(z\leq 0)=0.25+0.5=0.75$\vskip0.5cm
                $-z_2 \simeq 0.67$\vskip0.5cm
                $z_2 \simeq -0.67$\vskip0.5cm
                $x_2=-0.67*7.9+74\simeq 68.7$

        \end{enumerate}\vskip1cm

    \textbf{14. }Las estaturas de 1000 estudiantes están normalmente distribuidas con una media de $174.5$ centímetros y
    una desviación estándar de $6.9$ centímetros. Suponiendo  que las alturas se registran cerrando los valores a los mdios
    centímetros, ¿cuántos estudiantes tendrían estaturas
        \begin{enumerate}[a)]
            \item menores que $160.0$ centímetros?\vskip0.5cm
                $z=\displaystyle\frac{160-174.5}{6.9}=-2.1$\vskip0.5cm
                $p(z\leq -2.1)=0.0179$
            \item entre $171.5$ y $182.0$ centímetros inclusive?\vskip0.5cm
                $z_1=\displaystyle\frac{171.5-174.5}{6.9}=-0.43$\vskip0.5cm
                $z_2=\displaystyle\frac{182-174.5}{6.9}=1.08$\vskip0.5cm
                $p(-0.43\leq z\leq 1.08)=p(z\leq 1.08)-p(z\leq -.43)=0.8599-0.3336=0.5254$
            \item de $175.0$ centímetros?\vskip0.5cm
                $z_1=\displaystyle\frac{174.5-174.5}{6.9}=0$\vskip0.5cm
                $z_2=\displaystyle\frac{175.5-174.5}{6.9}=0.14$\vskip0.5cm
                $p(0\leq z\leq 0.14)=p(z\leq 0.14)-p(z\leq 0)=0.5557-0.5=0.0557$
            \item mayores que o iguales a $188.0$ centímetros?\vskip0.5cm
                $z=\displaystyle\frac{188-174.5}{6.9}=1.95$\vskip0.5cm
                $p(z\geq 1.95)=p(z\leq -1.95)=0.256$
        \end{enumerate}\vskip1cm

    \textbf{16. }Los pesos de un número grande de perros de lana miniatura están dstribuidos aproximadamente de forma normal
    con una media de $8$ kilogramos y  una desviación estándar de $0.9$ kilogramos. Si se registran mediciones y se cierran
    a décimas de kilogramo, encuentre la fracción de éstos perros de lana con pesos
        \begin{enumerate}[a)]
            \item arriba de $9.5$ kilogramos\vskip0.5cm
                $z=\displaystyle\frac{9.5-8}{0.9}=1.6$\vskip0.5cm
                $p(z\geq 1.6)=p(z\leq -1.6)=0.548$
            \item cuando mucho $8.6$ kilogramos\vskip0.5cm
                $z=\displaystyle\frac{8.6-8}{0.9}=0.66$\vskip0.5cm
                $p(\leq \leq 0.66) = 0.7454$
            \item entre $7.3$ y $9.1$ kilogramos inclusive\vskip0.5cm
                $z_1=\displaystyle\frac{7.3-8}{0.9}=-0.77$\vskip0.5cm
                $z_2=\displaystyle\frac{9.1-8}{0.9}=1.22$\vskip0.5cm
                $p(-0.77\leq \leq \leq 1.22)=p(z\leq 1.22)-p(\leq \leq -0.77)=0.8888-0.2206=0.6682$
        \end{enumerate}\vskip1cm

    \textbf{18. }Si un conjunto de observaciones están normalmente distribuidas, ¿qué porcentaje de éstas difiere de la
    media en
        \begin{enumerate}[a)]
            \item más de $1.3\sigma$?\vskip0.5cm
                $p(z\geq 1.3)=p(z\leq -1.3)=0.968$
            \item menos de $0.52\sigma$?\vskip0.5cm
                $p(z\leq 0.52)=0.6985$
        \end{enumerate}\vskip1cm

    \textbf{20. }La precipiación pluvial promedio, registrada hasta centécimas de milímetro, en Roanoke, Virginia, en el
    mes de Marzo es de $9.22$ centímetros. Suponiendo que se trata de una distribución normal con una desviación estándar
    $2.83$ centímetros, encuentre la probabilidad de que el próximo marzo Roanoke tenga
        \begin{enumerate}[a)]
            \item menos de $1.84$ centímetros de lluvia\vskip0.5cm
                $z=\displaystyle\frac{1.84-9.22}{2.83}=-2.6$\vskip0.5cm
                $p(z\leq -2.6)=0.0047$
            \item más de $5$ centímetros pero no más de 7 de lluvia\vskip0.5cm
                $z=\displaystyle\frac{5-9.22}{2.83}=-1.49$\vskip0.5cm
                $p(z\geq -1.49)=p(z\leq 1.49)=0.9319$
            \item más de $13.8$ centímetros de lluvia\vskip0.5cm
                $z=\displaystyle\frac{13.8-9.22}{2.83}=1.61$\vskip0.5cm
                $p(z\geq 1.61)=p(z\leq -1.61)=0.537$
        \end{enumerate}\vskip1cm

    \textbf{22. }Una moneda se lanza 400 veces. Utilice la aproximación de la curva normal para encontrar la probabilidad
    de obtener
        \begin{enumerate}[a)]
            \item entre $185$ y $210$ caras inclusive\vskip0.5cm
                $\mu=0.5*400=200$; $\sigma=\sqrt{0.5*0.5*400}=\sqrt{100}=10$\vskip0.5cm
                $z_1=\displaystyle\frac{185-200}{10}=-1.5$\vskip0.5cm
                $z_2=\displaystyle\frac{210-200}{10}=1$\vskip0.5cm
                $p(-1.5\leq z\leq 1)=p(z\leq 1)-p(z\leq -1.5)=0.8413-0.0668=0.7745$
            \item exactamente $205$ caras\vskip0.5cm
                $z_1=\displaystyle\frac{204.5-200}{10}=0.45$\vskip0.5cm
                $z_2=\displaystyle\frac{205.5-200}{10}=0.55$\vskip0.5cm
                $p(0.45\leq z\leq 0.55)=p(z\leq 0.55)-p(z\leq 0.45)=0.7088-0.6736=0.0352$
            \item menos de $176$ o más de $227$ caras\vskip0.5cm
                $z_1=\displaystyle\frac{176-200}{10}=-2.4$\vskip0.5cm
                $z_2=\displaystyle\frac{227-200}{10}=2.7$\vskip0.5cm
                $p(-2.4\leq z\leq 2.7)=p(z\leq 2.7)-p(z\leq -2.4)=0.9965-0.0082=0.9878$
        \end{enumerate}\vskip1cm

    \textbf{24. }Un proceso produce un $10\%$ de artículos defectuosos. Si se seleccionan del proceso 100 artículos
    aleatoriamente, ¿cuál es la probabilidad de que el número de defectuosos
        \begin{enumerate}[a)]
            \item exceda de $13$?\vskip0.5cm
                $\mu = 100*0.1=10$; $\sigma = \sqrt{100*0.1*0.9}=\sqrt{9}=3$\vskip0.5cm
                $z=\displaystyle\frac{13-10}{3}=1$\vskip0.5cm
                $p(1\geq z)=p(z\leq -1)=0.1587$
            \item sea menor de $8$?\vskip0.5cm
                $z=\displaystyle\frac{8-10}{9}=-0.66$\vskip0.5cm
                $p(z\leq -0.66)=0.2546$
        \end{enumerate}\vskip1cm

    \textbf{26. }Investigadores de la George Washington University y el National Institute of Helath reportan que
    aproximadamente $75\%$ de las personas creen que "los tranquilizantes funcionan muy bien para que una persona esté
    más tranquila y relajada". De las siguientes 80 personas entrevistadas, ¿cuál es la probabilidad de que
        \begin{enumerate}[a)]
            \item al menos 50 sean de la misma opinión?\vskip0.5cm
                $\mu = 80*0.75=60$; $\sigma = \sqrt{80*.75*.25}=\sqrt{15}=3.87$\vskip0.5cm
                $z=\displaystyle\frac{50-60}{3.87}=-2.58$\vskip0.5cm
                $p(-2.58\geq z)=p(z\leq 2.58)=0.9951$\vskip0.5cm
            \item más de 56 sean de la misma opinión?\vskip0.5cm
                $z=\displaystyle\frac{56-60}{3.87}=-1.03$\vskip0.5cm
                $p(z\geq -1.03)=p(z\leq 1.03)=0.8485$
        \end{enumerate}\vskip1cm

    \textbf{28. }Un fabricante de medicamentos sostiene que cierta medicina cura una enfermedad de la sangre en el $80\%$
    de los casos. Para verificarlo, los ispectores del gobierno utilizan el medicamento en una muestra de 100 individuos y
    deciden aceptar dicha afirmación si se curan 75 o más
        \begin{enumerate}[a)]
            \item ¿Cuál es la probabilidad de que lo que se dice sea rechazado cuando la probabilidad de curación sea,
                  en efecto, 0.8?\vskip0.5cm
                $\mu = 100*0.8=80$; $\sigma = \sqrt{100*.80*.20}=\sqrt{16}=4$\vskip0.5cm
                $z=\displaystyle\frac{75-80}{4}=-1.25$\vskip0.5cm
                $p(z\geq -1.25)=p(z\leq 1.25)=0.8944$
            \item ¿Cuál es la probabilidad de que la afirmación sea aceptada por el gobierno cuando la probabilidad de
                  curación sea menor a 0.7\vskip0.5cm
                $\mu=100*0.7=70$; $\sigma = \sqrt{100*0.7*0.3}=\sqrt{21}=4.58$\vskip0.5cm
                $z=\displaystyle\frac{75-70}{4.58}=1.09$\vskip0.5cm
                $p(z\geq 1.09)=p(z\leq -1.09)=0.1379$
        \end{enumerate}\vskip1cm

    \textbf{30. }Una compañia farmacéutica sabe que aproximadaente $5\%$ de sus píldoras para el control natal tiene un
    ingrediente que está por debajo de la dosis mínima, lo que vuelve ineficaz la píldora. ¿Cuál es la probabilidad de que
    menos de 10 en una muestra de 200 sea ineficaz?\vskip0.5cm
        $\mu = 200*0.05=10$; $\sigma = \sqrt{200*.05*.95}=\sqrt{9.5}=3.08$\vskip0.5cm
        $z=\displaystyle\frac{10-10}{3.08}=0$\vskip0.5cm
        $p(z\leq 0)=0.5$

\end{document}